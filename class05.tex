% Options for packages loaded elsewhere
\PassOptionsToPackage{unicode}{hyperref}
\PassOptionsToPackage{hyphens}{url}
%
\documentclass[
]{article}
\title{class05.R}
\author{meggg}
\date{2022-02-05}

\usepackage{amsmath,amssymb}
\usepackage{lmodern}
\usepackage{iftex}
\ifPDFTeX
  \usepackage[T1]{fontenc}
  \usepackage[utf8]{inputenc}
  \usepackage{textcomp} % provide euro and other symbols
\else % if luatex or xetex
  \usepackage{unicode-math}
  \defaultfontfeatures{Scale=MatchLowercase}
  \defaultfontfeatures[\rmfamily]{Ligatures=TeX,Scale=1}
\fi
% Use upquote if available, for straight quotes in verbatim environments
\IfFileExists{upquote.sty}{\usepackage{upquote}}{}
\IfFileExists{microtype.sty}{% use microtype if available
  \usepackage[]{microtype}
  \UseMicrotypeSet[protrusion]{basicmath} % disable protrusion for tt fonts
}{}
\makeatletter
\@ifundefined{KOMAClassName}{% if non-KOMA class
  \IfFileExists{parskip.sty}{%
    \usepackage{parskip}
  }{% else
    \setlength{\parindent}{0pt}
    \setlength{\parskip}{6pt plus 2pt minus 1pt}}
}{% if KOMA class
  \KOMAoptions{parskip=half}}
\makeatother
\usepackage{xcolor}
\IfFileExists{xurl.sty}{\usepackage{xurl}}{} % add URL line breaks if available
\IfFileExists{bookmark.sty}{\usepackage{bookmark}}{\usepackage{hyperref}}
\hypersetup{
  pdftitle={class05.R},
  pdfauthor={meggg},
  hidelinks,
  pdfcreator={LaTeX via pandoc}}
\urlstyle{same} % disable monospaced font for URLs
\usepackage[margin=1in]{geometry}
\usepackage{color}
\usepackage{fancyvrb}
\newcommand{\VerbBar}{|}
\newcommand{\VERB}{\Verb[commandchars=\\\{\}]}
\DefineVerbatimEnvironment{Highlighting}{Verbatim}{commandchars=\\\{\}}
% Add ',fontsize=\small' for more characters per line
\usepackage{framed}
\definecolor{shadecolor}{RGB}{248,248,248}
\newenvironment{Shaded}{\begin{snugshade}}{\end{snugshade}}
\newcommand{\AlertTok}[1]{\textcolor[rgb]{0.94,0.16,0.16}{#1}}
\newcommand{\AnnotationTok}[1]{\textcolor[rgb]{0.56,0.35,0.01}{\textbf{\textit{#1}}}}
\newcommand{\AttributeTok}[1]{\textcolor[rgb]{0.77,0.63,0.00}{#1}}
\newcommand{\BaseNTok}[1]{\textcolor[rgb]{0.00,0.00,0.81}{#1}}
\newcommand{\BuiltInTok}[1]{#1}
\newcommand{\CharTok}[1]{\textcolor[rgb]{0.31,0.60,0.02}{#1}}
\newcommand{\CommentTok}[1]{\textcolor[rgb]{0.56,0.35,0.01}{\textit{#1}}}
\newcommand{\CommentVarTok}[1]{\textcolor[rgb]{0.56,0.35,0.01}{\textbf{\textit{#1}}}}
\newcommand{\ConstantTok}[1]{\textcolor[rgb]{0.00,0.00,0.00}{#1}}
\newcommand{\ControlFlowTok}[1]{\textcolor[rgb]{0.13,0.29,0.53}{\textbf{#1}}}
\newcommand{\DataTypeTok}[1]{\textcolor[rgb]{0.13,0.29,0.53}{#1}}
\newcommand{\DecValTok}[1]{\textcolor[rgb]{0.00,0.00,0.81}{#1}}
\newcommand{\DocumentationTok}[1]{\textcolor[rgb]{0.56,0.35,0.01}{\textbf{\textit{#1}}}}
\newcommand{\ErrorTok}[1]{\textcolor[rgb]{0.64,0.00,0.00}{\textbf{#1}}}
\newcommand{\ExtensionTok}[1]{#1}
\newcommand{\FloatTok}[1]{\textcolor[rgb]{0.00,0.00,0.81}{#1}}
\newcommand{\FunctionTok}[1]{\textcolor[rgb]{0.00,0.00,0.00}{#1}}
\newcommand{\ImportTok}[1]{#1}
\newcommand{\InformationTok}[1]{\textcolor[rgb]{0.56,0.35,0.01}{\textbf{\textit{#1}}}}
\newcommand{\KeywordTok}[1]{\textcolor[rgb]{0.13,0.29,0.53}{\textbf{#1}}}
\newcommand{\NormalTok}[1]{#1}
\newcommand{\OperatorTok}[1]{\textcolor[rgb]{0.81,0.36,0.00}{\textbf{#1}}}
\newcommand{\OtherTok}[1]{\textcolor[rgb]{0.56,0.35,0.01}{#1}}
\newcommand{\PreprocessorTok}[1]{\textcolor[rgb]{0.56,0.35,0.01}{\textit{#1}}}
\newcommand{\RegionMarkerTok}[1]{#1}
\newcommand{\SpecialCharTok}[1]{\textcolor[rgb]{0.00,0.00,0.00}{#1}}
\newcommand{\SpecialStringTok}[1]{\textcolor[rgb]{0.31,0.60,0.02}{#1}}
\newcommand{\StringTok}[1]{\textcolor[rgb]{0.31,0.60,0.02}{#1}}
\newcommand{\VariableTok}[1]{\textcolor[rgb]{0.00,0.00,0.00}{#1}}
\newcommand{\VerbatimStringTok}[1]{\textcolor[rgb]{0.31,0.60,0.02}{#1}}
\newcommand{\WarningTok}[1]{\textcolor[rgb]{0.56,0.35,0.01}{\textbf{\textit{#1}}}}
\usepackage{graphicx}
\makeatletter
\def\maxwidth{\ifdim\Gin@nat@width>\linewidth\linewidth\else\Gin@nat@width\fi}
\def\maxheight{\ifdim\Gin@nat@height>\textheight\textheight\else\Gin@nat@height\fi}
\makeatother
% Scale images if necessary, so that they will not overflow the page
% margins by default, and it is still possible to overwrite the defaults
% using explicit options in \includegraphics[width, height, ...]{}
\setkeys{Gin}{width=\maxwidth,height=\maxheight,keepaspectratio}
% Set default figure placement to htbp
\makeatletter
\def\fps@figure{htbp}
\makeatother
\setlength{\emergencystretch}{3em} % prevent overfull lines
\providecommand{\tightlist}{%
  \setlength{\itemsep}{0pt}\setlength{\parskip}{0pt}}
\setcounter{secnumdepth}{-\maxdimen} % remove section numbering
\ifLuaTeX
  \usepackage{selnolig}  % disable illegal ligatures
\fi

\begin{document}
\maketitle

\begin{Shaded}
\begin{Highlighting}[]
\DocumentationTok{\#\#\#\#\#\#\#\#\#\#\#\#\#\#\#\#\#\#\#\#\#\#\#\#\#\#\#\#\#\#\#\#\#\#\#\#}
\CommentTok{\# Author: Meg Robinson}
\CommentTok{\# Class 5}
\CommentTok{\# Section 5: Scatter Plots}
\DocumentationTok{\#\#\#\#\#\#\#\#\#\#\#\#\#\#\#\#\#\#\#\#\#\#\#\#\#\#\#\#\#\#\#\#\#\#\#\#}


\CommentTok{\# load the ggplot2 package after installation}
\FunctionTok{library}\NormalTok{(ggplot2)}

\CommentTok{\# Use the built{-}in \textquotesingle{}cars\textquotesingle{} dataset for visualization }

\CommentTok{\# Look at whats in the dataset}
\FunctionTok{head}\NormalTok{(cars)}
\end{Highlighting}
\end{Shaded}

\begin{verbatim}
##   speed dist
## 1     4    2
## 2     4   10
## 3     7    4
## 4     7   22
## 5     8   16
## 6     9   10
\end{verbatim}

\begin{Shaded}
\begin{Highlighting}[]
\CommentTok{\# Use ggplots 3 layers: data, aes, geoms}
\CommentTok{\# To visualize \textquotesingle{}cars\textquotesingle{} as a scatterplot}
\CommentTok{\# Use + to add layers}
\NormalTok{plot }\OtherTok{=} \FunctionTok{ggplot}\NormalTok{(cars) }\SpecialCharTok{+}
  \FunctionTok{aes}\NormalTok{(}\AttributeTok{x=}\NormalTok{speed, }\AttributeTok{y=}\NormalTok{dist) }\SpecialCharTok{+} \FunctionTok{geom\_point}\NormalTok{() }
\NormalTok{plot}
\end{Highlighting}
\end{Shaded}

\includegraphics{class05_files/figure-latex/unnamed-chunk-1-1.pdf}

\begin{Shaded}
\begin{Highlighting}[]
\CommentTok{\# Add geom\_smooth() to show the relationship between variables}
\NormalTok{plot }\OtherTok{=}\NormalTok{ plot }\SpecialCharTok{+} \FunctionTok{geom\_smooth}\NormalTok{(}\AttributeTok{method=}\StringTok{\textquotesingle{}lm\textquotesingle{}}\NormalTok{, }\AttributeTok{se=}\ConstantTok{FALSE}\NormalTok{)}
\NormalTok{plot}
\end{Highlighting}
\end{Shaded}

\begin{verbatim}
## `geom_smooth()` using formula 'y ~ x'
\end{verbatim}

\includegraphics{class05_files/figure-latex/unnamed-chunk-1-2.pdf}

\begin{Shaded}
\begin{Highlighting}[]
\NormalTok{plot }\OtherTok{=}\NormalTok{ plot }\SpecialCharTok{+} \FunctionTok{labs}\NormalTok{(}\AttributeTok{title=}\StringTok{\textquotesingle{}Speed vs Distance of Cars\textquotesingle{}}\NormalTok{, }
                   \AttributeTok{x=}\StringTok{"Speed (mph)"}\NormalTok{,}
                   \AttributeTok{y=}\StringTok{"Distance to stop (ft)"}\NormalTok{)  }\SpecialCharTok{+}
  \FunctionTok{theme\_bw}\NormalTok{()}
\NormalTok{plot}
\end{Highlighting}
\end{Shaded}

\begin{verbatim}
## `geom_smooth()` using formula 'y ~ x'
\end{verbatim}

\includegraphics{class05_files/figure-latex/unnamed-chunk-1-3.pdf}

\begin{Shaded}
\begin{Highlighting}[]
\CommentTok{\# Read in the results of DE analysis}
\NormalTok{url }\OtherTok{\textless{}{-}} \StringTok{"https://bioboot.github.io/bimm143\_S20/class{-}material/up\_down\_expression.txt"}
\NormalTok{genes }\OtherTok{\textless{}{-}} \FunctionTok{read.delim}\NormalTok{(url)}
\FunctionTok{head}\NormalTok{(genes)}
\end{Highlighting}
\end{Shaded}

\begin{verbatim}
##         Gene Condition1 Condition2      State
## 1      A4GNT -3.6808610 -3.4401355 unchanging
## 2       AAAS  4.5479580  4.3864126 unchanging
## 3      AASDH  3.7190695  3.4787276 unchanging
## 4       AATF  5.0784720  5.0151916 unchanging
## 5       AATK  0.4711421  0.5598642 unchanging
## 6 AB015752.4 -3.6808610 -3.5921390 unchanging
\end{verbatim}

\begin{Shaded}
\begin{Highlighting}[]
\CommentTok{\# How many genes are in the dataset?}
\FunctionTok{nrow}\NormalTok{(genes)}
\end{Highlighting}
\end{Shaded}

\begin{verbatim}
## [1] 5196
\end{verbatim}

\begin{Shaded}
\begin{Highlighting}[]
\CommentTok{\# How many columns are in the dataset?}
\FunctionTok{ncol}\NormalTok{(genes)}
\end{Highlighting}
\end{Shaded}

\begin{verbatim}
## [1] 4
\end{verbatim}

\begin{Shaded}
\begin{Highlighting}[]
\CommentTok{\# What are the column names?}
\FunctionTok{colnames}\NormalTok{(genes)}
\end{Highlighting}
\end{Shaded}

\begin{verbatim}
## [1] "Gene"       "Condition1" "Condition2" "State"
\end{verbatim}

\begin{Shaded}
\begin{Highlighting}[]
\CommentTok{\# How many upregulated genes are there?}
\FunctionTok{table}\NormalTok{(genes}\SpecialCharTok{$}\NormalTok{State)}
\end{Highlighting}
\end{Shaded}

\begin{verbatim}
## 
##       down unchanging         up 
##         72       4997        127
\end{verbatim}

\begin{Shaded}
\begin{Highlighting}[]
\CommentTok{\# What percent of up regulated genes?}
\FunctionTok{round}\NormalTok{(}\FunctionTok{table}\NormalTok{(genes}\SpecialCharTok{$}\NormalTok{State)}\SpecialCharTok{/}\FunctionTok{nrow}\NormalTok{(genes)}\SpecialCharTok{*}\DecValTok{100}\NormalTok{,}\DecValTok{2}\NormalTok{)}
\end{Highlighting}
\end{Shaded}

\begin{verbatim}
## 
##       down unchanging         up 
##       1.39      96.17       2.44
\end{verbatim}

\begin{Shaded}
\begin{Highlighting}[]
\CommentTok{\# Make a scatter plot of the genes df}
\FunctionTok{ggplot}\NormalTok{(genes) }\SpecialCharTok{+} 
  \FunctionTok{aes}\NormalTok{(}\AttributeTok{x=}\NormalTok{Condition1, }\AttributeTok{y=}\NormalTok{Condition2) }\SpecialCharTok{+}
  \FunctionTok{geom\_point}\NormalTok{()}
\end{Highlighting}
\end{Shaded}

\includegraphics{class05_files/figure-latex/unnamed-chunk-1-4.pdf}

\begin{Shaded}
\begin{Highlighting}[]
\CommentTok{\# Map the the column  State to color}
\NormalTok{gene\_plot }\OtherTok{=} \FunctionTok{ggplot}\NormalTok{(genes) }\SpecialCharTok{+} 
  \FunctionTok{aes}\NormalTok{(}\AttributeTok{x=}\NormalTok{Condition1, }\AttributeTok{y=}\NormalTok{Condition2, }\AttributeTok{col=}\NormalTok{State) }\SpecialCharTok{+}
  \FunctionTok{geom\_point}\NormalTok{()}
\NormalTok{gene\_plot}
\end{Highlighting}
\end{Shaded}

\includegraphics{class05_files/figure-latex/unnamed-chunk-1-5.pdf}

\begin{Shaded}
\begin{Highlighting}[]
\CommentTok{\# add titles etc}
\NormalTok{gene\_plot }\OtherTok{=}\NormalTok{ gene\_plot }\SpecialCharTok{+} \FunctionTok{theme\_bw}\NormalTok{() }\SpecialCharTok{+} \FunctionTok{scale\_colour\_manual}\NormalTok{(}\AttributeTok{values=}\FunctionTok{c}\NormalTok{(}\StringTok{"blue"}\NormalTok{,}\StringTok{"gray"}\NormalTok{,}\StringTok{"red"}\NormalTok{)) }\SpecialCharTok{+}
  \FunctionTok{labs}\NormalTok{(}\AttributeTok{title=}\StringTok{"Gene Expresion Changes Upon Drug Treatment"}\NormalTok{,}
       \AttributeTok{x=}\StringTok{"Control (no drug) "}\NormalTok{,}
       \AttributeTok{y=}\StringTok{"Drug Treatment"}\NormalTok{)}
\NormalTok{gene\_plot}
\end{Highlighting}
\end{Shaded}

\includegraphics{class05_files/figure-latex/unnamed-chunk-1-6.pdf}

\end{document}
