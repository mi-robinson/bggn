% Options for packages loaded elsewhere
\PassOptionsToPackage{unicode}{hyperref}
\PassOptionsToPackage{hyphens}{url}
%
\documentclass[
]{article}
\title{Week 5: R Functions}
\author{Meg Robinson}
\date{2/5/2022}

\usepackage{amsmath,amssymb}
\usepackage{lmodern}
\usepackage{iftex}
\ifPDFTeX
  \usepackage[T1]{fontenc}
  \usepackage[utf8]{inputenc}
  \usepackage{textcomp} % provide euro and other symbols
\else % if luatex or xetex
  \usepackage{unicode-math}
  \defaultfontfeatures{Scale=MatchLowercase}
  \defaultfontfeatures[\rmfamily]{Ligatures=TeX,Scale=1}
\fi
% Use upquote if available, for straight quotes in verbatim environments
\IfFileExists{upquote.sty}{\usepackage{upquote}}{}
\IfFileExists{microtype.sty}{% use microtype if available
  \usepackage[]{microtype}
  \UseMicrotypeSet[protrusion]{basicmath} % disable protrusion for tt fonts
}{}
\makeatletter
\@ifundefined{KOMAClassName}{% if non-KOMA class
  \IfFileExists{parskip.sty}{%
    \usepackage{parskip}
  }{% else
    \setlength{\parindent}{0pt}
    \setlength{\parskip}{6pt plus 2pt minus 1pt}}
}{% if KOMA class
  \KOMAoptions{parskip=half}}
\makeatother
\usepackage{xcolor}
\IfFileExists{xurl.sty}{\usepackage{xurl}}{} % add URL line breaks if available
\IfFileExists{bookmark.sty}{\usepackage{bookmark}}{\usepackage{hyperref}}
\hypersetup{
  pdftitle={Week 5: R Functions},
  pdfauthor={Meg Robinson},
  hidelinks,
  pdfcreator={LaTeX via pandoc}}
\urlstyle{same} % disable monospaced font for URLs
\usepackage[margin=1in]{geometry}
\usepackage{color}
\usepackage{fancyvrb}
\newcommand{\VerbBar}{|}
\newcommand{\VERB}{\Verb[commandchars=\\\{\}]}
\DefineVerbatimEnvironment{Highlighting}{Verbatim}{commandchars=\\\{\}}
% Add ',fontsize=\small' for more characters per line
\usepackage{framed}
\definecolor{shadecolor}{RGB}{248,248,248}
\newenvironment{Shaded}{\begin{snugshade}}{\end{snugshade}}
\newcommand{\AlertTok}[1]{\textcolor[rgb]{0.94,0.16,0.16}{#1}}
\newcommand{\AnnotationTok}[1]{\textcolor[rgb]{0.56,0.35,0.01}{\textbf{\textit{#1}}}}
\newcommand{\AttributeTok}[1]{\textcolor[rgb]{0.77,0.63,0.00}{#1}}
\newcommand{\BaseNTok}[1]{\textcolor[rgb]{0.00,0.00,0.81}{#1}}
\newcommand{\BuiltInTok}[1]{#1}
\newcommand{\CharTok}[1]{\textcolor[rgb]{0.31,0.60,0.02}{#1}}
\newcommand{\CommentTok}[1]{\textcolor[rgb]{0.56,0.35,0.01}{\textit{#1}}}
\newcommand{\CommentVarTok}[1]{\textcolor[rgb]{0.56,0.35,0.01}{\textbf{\textit{#1}}}}
\newcommand{\ConstantTok}[1]{\textcolor[rgb]{0.00,0.00,0.00}{#1}}
\newcommand{\ControlFlowTok}[1]{\textcolor[rgb]{0.13,0.29,0.53}{\textbf{#1}}}
\newcommand{\DataTypeTok}[1]{\textcolor[rgb]{0.13,0.29,0.53}{#1}}
\newcommand{\DecValTok}[1]{\textcolor[rgb]{0.00,0.00,0.81}{#1}}
\newcommand{\DocumentationTok}[1]{\textcolor[rgb]{0.56,0.35,0.01}{\textbf{\textit{#1}}}}
\newcommand{\ErrorTok}[1]{\textcolor[rgb]{0.64,0.00,0.00}{\textbf{#1}}}
\newcommand{\ExtensionTok}[1]{#1}
\newcommand{\FloatTok}[1]{\textcolor[rgb]{0.00,0.00,0.81}{#1}}
\newcommand{\FunctionTok}[1]{\textcolor[rgb]{0.00,0.00,0.00}{#1}}
\newcommand{\ImportTok}[1]{#1}
\newcommand{\InformationTok}[1]{\textcolor[rgb]{0.56,0.35,0.01}{\textbf{\textit{#1}}}}
\newcommand{\KeywordTok}[1]{\textcolor[rgb]{0.13,0.29,0.53}{\textbf{#1}}}
\newcommand{\NormalTok}[1]{#1}
\newcommand{\OperatorTok}[1]{\textcolor[rgb]{0.81,0.36,0.00}{\textbf{#1}}}
\newcommand{\OtherTok}[1]{\textcolor[rgb]{0.56,0.35,0.01}{#1}}
\newcommand{\PreprocessorTok}[1]{\textcolor[rgb]{0.56,0.35,0.01}{\textit{#1}}}
\newcommand{\RegionMarkerTok}[1]{#1}
\newcommand{\SpecialCharTok}[1]{\textcolor[rgb]{0.00,0.00,0.00}{#1}}
\newcommand{\SpecialStringTok}[1]{\textcolor[rgb]{0.31,0.60,0.02}{#1}}
\newcommand{\StringTok}[1]{\textcolor[rgb]{0.31,0.60,0.02}{#1}}
\newcommand{\VariableTok}[1]{\textcolor[rgb]{0.00,0.00,0.00}{#1}}
\newcommand{\VerbatimStringTok}[1]{\textcolor[rgb]{0.31,0.60,0.02}{#1}}
\newcommand{\WarningTok}[1]{\textcolor[rgb]{0.56,0.35,0.01}{\textbf{\textit{#1}}}}
\usepackage{graphicx}
\makeatletter
\def\maxwidth{\ifdim\Gin@nat@width>\linewidth\linewidth\else\Gin@nat@width\fi}
\def\maxheight{\ifdim\Gin@nat@height>\textheight\textheight\else\Gin@nat@height\fi}
\makeatother
% Scale images if necessary, so that they will not overflow the page
% margins by default, and it is still possible to overwrite the defaults
% using explicit options in \includegraphics[width, height, ...]{}
\setkeys{Gin}{width=\maxwidth,height=\maxheight,keepaspectratio}
% Set default figure placement to htbp
\makeatletter
\def\fps@figure{htbp}
\makeatother
\setlength{\emergencystretch}{3em} % prevent overfull lines
\providecommand{\tightlist}{%
  \setlength{\itemsep}{0pt}\setlength{\parskip}{0pt}}
\setcounter{secnumdepth}{-\maxdimen} % remove section numbering
\ifLuaTeX
  \usepackage{selnolig}  % disable illegal ligatures
\fi

\begin{document}
\maketitle

\begin{quote}
Q1. Write a function grade() to determine an overall grade from a vector
of student homework assignment scores dropping the lowest single score.
If a student misses a homework (i.e.~has an NA value) this can be used
as a score to be potentially dropped. Your final function should be
adquately explained with code comments and be able to work on an example
class gradebook such as this one in CSV format:
``\url{https://tinyurl.com/gradeinput}'' {[}3pts{]}
\end{quote}

\begin{Shaded}
\begin{Highlighting}[]
\CommentTok{\# Example input vectors to start with}
\NormalTok{student1 }\OtherTok{\textless{}{-}} \FunctionTok{c}\NormalTok{(}\DecValTok{100}\NormalTok{, }\DecValTok{100}\NormalTok{, }\DecValTok{100}\NormalTok{, }\DecValTok{100}\NormalTok{, }\DecValTok{100}\NormalTok{, }\DecValTok{100}\NormalTok{, }\DecValTok{100}\NormalTok{, }\DecValTok{90}\NormalTok{)}
\NormalTok{student2 }\OtherTok{\textless{}{-}} \FunctionTok{c}\NormalTok{(}\DecValTok{100}\NormalTok{, }\ConstantTok{NA}\NormalTok{, }\DecValTok{90}\NormalTok{, }\DecValTok{90}\NormalTok{, }\DecValTok{90}\NormalTok{, }\DecValTok{90}\NormalTok{, }\DecValTok{97}\NormalTok{, }\DecValTok{80}\NormalTok{)}
\NormalTok{student3 }\OtherTok{\textless{}{-}} \FunctionTok{c}\NormalTok{(}\DecValTok{90}\NormalTok{, }\ConstantTok{NA}\NormalTok{, }\ConstantTok{NA}\NormalTok{, }\ConstantTok{NA}\NormalTok{, }\ConstantTok{NA}\NormalTok{, }\ConstantTok{NA}\NormalTok{, }\ConstantTok{NA}\NormalTok{, }\ConstantTok{NA}\NormalTok{)}
\end{Highlighting}
\end{Shaded}

Average of student 1

\begin{Shaded}
\begin{Highlighting}[]
\NormalTok{student1}
\end{Highlighting}
\end{Shaded}

\begin{verbatim}
## [1] 100 100 100 100 100 100 100  90
\end{verbatim}

\begin{Shaded}
\begin{Highlighting}[]
\FunctionTok{mean}\NormalTok{(student1)}
\end{Highlighting}
\end{Shaded}

\begin{verbatim}
## [1] 98.75
\end{verbatim}

Use `min()' to find lowest score

\begin{Shaded}
\begin{Highlighting}[]
\FunctionTok{min}\NormalTok{(student1)}
\end{Highlighting}
\end{Shaded}

\begin{verbatim}
## [1] 90
\end{verbatim}

Find index at which min occured using which.min()

\begin{Shaded}
\begin{Highlighting}[]
\FunctionTok{which.min}\NormalTok{(student1)}
\end{Highlighting}
\end{Shaded}

\begin{verbatim}
## [1] 8
\end{verbatim}

Get everything except lowest score using minus (``-'') and caclulate
mean (as long as no ``NA'' in vector)

\begin{Shaded}
\begin{Highlighting}[]
\FunctionTok{mean}\NormalTok{(student1[}\SpecialCharTok{{-}}\FunctionTok{which.min}\NormalTok{(student1)])}
\end{Highlighting}
\end{Shaded}

\begin{verbatim}
## [1] 100
\end{verbatim}

Try it on student 2

\begin{Shaded}
\begin{Highlighting}[]
\FunctionTok{mean}\NormalTok{(student2[}\SpecialCharTok{{-}}\FunctionTok{which.min}\NormalTok{(student1)])}
\end{Highlighting}
\end{Shaded}

\begin{verbatim}
## [1] NA
\end{verbatim}

It does not work because mean() function doesn't allow for \textbf{NA}
values. Now find \textbf{NA} values

\begin{Shaded}
\begin{Highlighting}[]
\FunctionTok{is.na}\NormalTok{(student2)}
\end{Highlighting}
\end{Shaded}

\begin{verbatim}
## [1] FALSE  TRUE FALSE FALSE FALSE FALSE FALSE FALSE
\end{verbatim}

Replace \emph{NA} with zero

\begin{Shaded}
\begin{Highlighting}[]
\NormalTok{student.prime }\OtherTok{=}\NormalTok{ student2}
\NormalTok{student.prime[}\FunctionTok{is.na}\NormalTok{(student.prime)] }\OtherTok{=} \DecValTok{0}
\NormalTok{student.prime}
\end{Highlighting}
\end{Shaded}

\begin{verbatim}
## [1] 100   0  90  90  90  90  97  80
\end{verbatim}

Now get mean()

\begin{Shaded}
\begin{Highlighting}[]
\FunctionTok{mean}\NormalTok{(student.prime[}\SpecialCharTok{{-}}\FunctionTok{which.min}\NormalTok{(student.prime)])}
\end{Highlighting}
\end{Shaded}

\begin{verbatim}
## [1] 91
\end{verbatim}

Which we can see is the value of \emph{student2}

\begin{Shaded}
\begin{Highlighting}[]
\FunctionTok{mean}\NormalTok{(}\FunctionTok{c}\NormalTok{(}\DecValTok{100}\NormalTok{,}\DecValTok{90}\NormalTok{,}\DecValTok{90}\NormalTok{,}\DecValTok{90}\NormalTok{,}\DecValTok{90}\NormalTok{,}\DecValTok{97}\NormalTok{,}\DecValTok{80}\NormalTok{))}
\end{Highlighting}
\end{Shaded}

\begin{verbatim}
## [1] 91
\end{verbatim}

So now do the above with \emph{student3}

\begin{Shaded}
\begin{Highlighting}[]
\NormalTok{x }\OtherTok{=}\NormalTok{ student3}
\NormalTok{x[}\FunctionTok{is.na}\NormalTok{(x)] }\OtherTok{=} \DecValTok{0}
\FunctionTok{mean}\NormalTok{(x[}\SpecialCharTok{{-}}\FunctionTok{which.min}\NormalTok{(x)])}
\end{Highlighting}
\end{Shaded}

\begin{verbatim}
## [1] 12.85714
\end{verbatim}

So we can write our function

\begin{Shaded}
\begin{Highlighting}[]
\CommentTok{\#\textquotesingle{} Calculate avg scores for a vector hw socres}
\CommentTok{\#\textquotesingle{} Drop lowest homework score}
\CommentTok{\#\textquotesingle{} Missing values treated as 0}
\CommentTok{\#\textquotesingle{}}
\CommentTok{\#\textquotesingle{} @param x Numeric vector of homework scores}
\CommentTok{\#\textquotesingle{}}
\CommentTok{\#\textquotesingle{} @return average score}
\CommentTok{\#\textquotesingle{} @export}
\CommentTok{\#\textquotesingle{}}
\CommentTok{\#\textquotesingle{} @examples}
\CommentTok{\#\textquotesingle{} student = c(100,NA,90, 80)}
\CommentTok{\#\textquotesingle{} grade(student)}
\CommentTok{\#\textquotesingle{} }
\NormalTok{grade }\OtherTok{=} \ControlFlowTok{function}\NormalTok{(x)\{}
  \CommentTok{\# Map NA missing hw vals to 0}
  \CommentTok{\# Assign hw scores 0}
\NormalTok{  x[}\FunctionTok{is.na}\NormalTok{(x)] }\OtherTok{=} \DecValTok{0}
  \CommentTok{\# Drop the lowest score}
  \FunctionTok{mean}\NormalTok{(x[}\SpecialCharTok{{-}}\FunctionTok{which.min}\NormalTok{(x)])}
\NormalTok{\}}
\end{Highlighting}
\end{Shaded}

And use it

\begin{Shaded}
\begin{Highlighting}[]
\FunctionTok{grade}\NormalTok{(student1)}
\end{Highlighting}
\end{Shaded}

\begin{verbatim}
## [1] 100
\end{verbatim}

\begin{Shaded}
\begin{Highlighting}[]
\FunctionTok{grade}\NormalTok{(student2)}
\end{Highlighting}
\end{Shaded}

\begin{verbatim}
## [1] 91
\end{verbatim}

\begin{Shaded}
\begin{Highlighting}[]
\FunctionTok{grade}\NormalTok{(student3)}
\end{Highlighting}
\end{Shaded}

\begin{verbatim}
## [1] 12.85714
\end{verbatim}

\hypertarget{now-grade-entire-class}{%
\subsubsection{Now grade entire class}\label{now-grade-entire-class}}

\begin{Shaded}
\begin{Highlighting}[]
\NormalTok{url }\OtherTok{=} \StringTok{"https://tinyurl.com/gradeinput"}
\NormalTok{gradebook }\OtherTok{=} \FunctionTok{read.csv}\NormalTok{(url, }\AttributeTok{row.names=}\DecValTok{1}\NormalTok{)}
\NormalTok{gradebook}
\end{Highlighting}
\end{Shaded}

\begin{verbatim}
##            hw1 hw2 hw3 hw4 hw5
## student-1  100  73 100  88  79
## student-2   85  64  78  89  78
## student-3   83  69  77 100  77
## student-4   88  NA  73 100  76
## student-5   88 100  75  86  79
## student-6   89  78 100  89  77
## student-7   89 100  74  87 100
## student-8   89 100  76  86 100
## student-9   86 100  77  88  77
## student-10  89  72  79  NA  76
## student-11  82  66  78  84 100
## student-12 100  70  75  92 100
## student-13  89 100  76 100  80
## student-14  85 100  77  89  76
## student-15  85  65  76  89  NA
## student-16  92 100  74  89  77
## student-17  88  63 100  86  78
## student-18  91  NA 100  87 100
## student-19  91  68  75  86  79
## student-20  91  68  76  88  76
\end{verbatim}

Use \textbf{apply()} to grade all of the students using our
\textbf{grade()} function

\begin{Shaded}
\begin{Highlighting}[]
\FunctionTok{apply}\NormalTok{(gradebook,}\DecValTok{1}\NormalTok{,grade)}
\end{Highlighting}
\end{Shaded}

\begin{verbatim}
##  student-1  student-2  student-3  student-4  student-5  student-6  student-7 
##      91.75      82.50      84.25      84.25      88.25      89.00      94.00 
##  student-8  student-9 student-10 student-11 student-12 student-13 student-14 
##      93.75      87.75      79.00      86.00      91.75      92.25      87.75 
## student-15 student-16 student-17 student-18 student-19 student-20 
##      78.75      89.50      88.00      94.50      82.75      82.75
\end{verbatim}

\begin{quote}
Q2. Using your grade() function and the supplied gradebook, Who is the
top scoring student overall in the gradebook? {[}3pts{]}
\end{quote}

\begin{Shaded}
\begin{Highlighting}[]
\NormalTok{finalgrades }\OtherTok{\textless{}{-}} \FunctionTok{apply}\NormalTok{(gradebook,}\DecValTok{1}\NormalTok{,grade)}
\FunctionTok{which.max}\NormalTok{(finalgrades)}
\end{Highlighting}
\end{Shaded}

\begin{verbatim}
## student-18 
##         18
\end{verbatim}

\begin{Shaded}
\begin{Highlighting}[]
\FunctionTok{max}\NormalTok{(finalgrades)}
\end{Highlighting}
\end{Shaded}

\begin{verbatim}
## [1] 94.5
\end{verbatim}

\textbf{student 18 is the top scoring student with an average score of
94.5}

\begin{quote}
Q3. From your analysis of the gradebook, which homework was toughest on
students (i.e.~obtained the lowest scores overall? {[}2pts
\end{quote}

Average hw's

\begin{Shaded}
\begin{Highlighting}[]
\NormalTok{hwavg }\OtherTok{=} \FunctionTok{apply}\NormalTok{(gradebook,}\DecValTok{2}\NormalTok{,mean, }\AttributeTok{na.rm=}\ConstantTok{TRUE}\NormalTok{)}
\FunctionTok{which.min}\NormalTok{(hwavg)}
\end{Highlighting}
\end{Shaded}

\begin{verbatim}
## hw3 
##   3
\end{verbatim}

Median of hw's

\begin{Shaded}
\begin{Highlighting}[]
\NormalTok{hwmed }\OtherTok{=} \FunctionTok{apply}\NormalTok{(gradebook,}\DecValTok{2}\NormalTok{,median, }\AttributeTok{na.rm=}\ConstantTok{TRUE}\NormalTok{)}
\FunctionTok{which.min}\NormalTok{(hwmed)}
\end{Highlighting}
\end{Shaded}

\begin{verbatim}
## hw2 
##   2
\end{verbatim}

Since results were different look at plot of gradebook

\begin{Shaded}
\begin{Highlighting}[]
\FunctionTok{boxplot}\NormalTok{(gradebook)}
\end{Highlighting}
\end{Shaded}

\includegraphics{class06_files/figure-latex/unnamed-chunk-19-1.pdf} It
looks like \textbf{homework 2} is the toughest.

\begin{quote}
Q4. Optional Extension: From your analysis of the gradebook, which
homework was most predictive of overall score (i.e.~highest correlation
with average grade score)? {[}1pt{]}
\end{quote}

Use the \textbf{cor()} function

\begin{Shaded}
\begin{Highlighting}[]
\NormalTok{gradebook[}\FunctionTok{is.na}\NormalTok{(gradebook)] }\OtherTok{=} \DecValTok{0}
\FunctionTok{cor}\NormalTok{(finalgrades, gradebook}\SpecialCharTok{$}\NormalTok{hw1)}
\end{Highlighting}
\end{Shaded}

\begin{verbatim}
## [1] 0.4250204
\end{verbatim}

\begin{Shaded}
\begin{Highlighting}[]
\FunctionTok{apply}\NormalTok{(gradebook, }\DecValTok{2}\NormalTok{, cor, }\AttributeTok{x=}\NormalTok{finalgrades)}
\end{Highlighting}
\end{Shaded}

\begin{verbatim}
##       hw1       hw2       hw3       hw4       hw5 
## 0.4250204 0.1767780 0.3042561 0.3810884 0.6325982
\end{verbatim}

\textbf{Homework 5} was most predictive of a studen't overall score

\end{document}
